% Chapter 1

\chapter{Introduction to the Thesis Topic} % Main chapter title

\label{Chapter1} % For referencing the chapter elsewhere, use \ref{Chapter1} 

%----------------------------------------------------------------------------------------

% Define some commands to keep the formatting separated from the content 
\newcommand{\keyword}[1]{\textbf{#1}}
\newcommand{\tabhead}[1]{\textbf{#1}}
\newcommand{\code}[1]{\texttt{#1}}
\newcommand{\file}[1]{\texttt{\bfseries#1}}
\newcommand{\option}[1]{\texttt{\itshape#1}}

%----------------------------------------------------------------------------------------
\section{Motivation and Background}
%goals
% - motivate topic models
% - explain what they are used for nowadays
% - explain how they typically work (distribution over topics)
% - describe how a dynamic model might improve on this
% - describe the usefulness of having a temporal component in the model. (maybe give specific example documents)
% - reference why applying it to patents is not an accident. i.e. why patents provide an interesting use case.

% What we explored:
% - how the word distributions of the topics trended (validate 
% with history)
% - Which models had the 'best' topics as measured by various topic coherence measures
% - Whether the resulting document vectors were useful for document classification
% - Whether the vector space resulted in quality clusters
% -- Usefulness as a measure of influence (f. citations and pagerank) 

% - If you believe all of these things you should give me a distinction. If not, read on...


% typical patent analysis usually includes forward citations as a measure of patent 'value' either economically or technologically.

%----------------------------------------------------------------------------------------

qualitative exploratory tasks and quantitative predictive and classification tasks.



%----------------------------------------------------------------------------------------


\section{examples}
\cite{Blei:2006:DTM:1143844.1143859}
\citep{Chang:Boyd-Graber:Wang:Gerrish:Blei-2009}
\citep{DBLP:journals/corr/RosnerHRNB14}
\citep{DBLP:journals/corr/abs-1206-3298}
\citep{Hall:2008:SHI:1613715.1613763}
\citep{Wang:2006:TOT:1150402.1150450}
\cite{conf/icdm/AlSumaitBD08}
\citep{icml2010_GerrishB10}

Multiple references are separated by semicolons (e.g. \parencite{Blei:2006:DTM:1143844.1143859, Wang:2006:TOT:1150402.1150450}) and 

Scientific references should come \emph{before} the punctuation mark if there is one (such as a comma or period). The same goes for footnotes\footnote{Such as this footnote, here down at the bottom of the page.}.states: \enquote{Footnote numbers should be superscripted, [...], following any punctuation mark except a dash.} The Chicago manual of style states: \enquote{A note number should be placed at the end of a sentence or clause. The number follows any punctuation mark except the dash, which it precedes. It follows a closing parenthesis.}

The bibliography is typeset with references listed in alphabetical order by the first author's last name. This is similar to the APA referencing style. To see how \LaTeX{} typesets the bibliography, have a look at the very end of this document (or just click on the reference number links in in-text citations).


Tables are an important way of displaying your results, below is an example table which was generated with this code:

{\small
\begin{verbatim}
\begin{table}
\caption{The effects of treatments X and Y on the four groups studied.}
\label{tab:treatments}
\centering
\begin{tabular}{l l l}
\toprule
\tabhead{Groups} & \tabhead{Treatment X} & \tabhead{Treatment Y} \\
\midrule
1 & 0.2 & 0.8\\
2 & 0.17 & 0.7\\
3 & 0.24 & 0.75\\
4 & 0.68 & 0.3\\
\bottomrule\\
\end{tabular}
\end{table}
\end{verbatim}
}

\begin{table}
\caption{The effects of treatments X and Y on the four groups studied.}
\label{tab:treatments}
\centering
\begin{tabular}{l l l}
\toprule
\tabhead{Groups} & \tabhead{Treatment X} & \tabhead{Treatment Y} \\
\midrule
1 & 0.2 & 0.8\\
2 & 0.17 & 0.7\\
3 & 0.24 & 0.75\\
4 & 0.68 & 0.3\\
\bottomrule\\
\end{tabular}
\end{table}

You can reference tables with \verb|\ref{<label>}| where the label is defined within the table environment. See \file{Chapter1.tex} for an example of the label and citation (e.g. Table~\ref{tab:treatments}).


\begin{figure}[h]
\centering
\includegraphics{Figures/Electron}
\decoRule
\caption[An Electron]{An electron (artist's impression).}
\label{fig:Electron}
\end{figure}

\begin{equation}
E = mc^{2}
\label{eqn:Einstein}
\end{equation}


\begin{flushright}
Guide written by ---\\
Sunil Patel: \href{http://www.sunilpatel.co.uk}{www.sunilpatel.co.uk}\\
Vel: \href{http://www.LaTeXTemplates.com}{LaTeXTemplates.com}
\end{flushright}
