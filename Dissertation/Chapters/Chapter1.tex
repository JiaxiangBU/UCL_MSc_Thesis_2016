% Chapter 1

\chapter{Introduction to the Thesis Topic} % Main chapter title

\label{Chapter1} % For referencing the chapter elsewhere, use \ref{Chapter1} 

%----------------------------------------------------------------------------------------

% Define some commands to keep the formatting separated from the content 
\newcommand{\keyword}[1]{\textbf{#1}}
\newcommand{\tabhead}[1]{\textbf{#1}}
\newcommand{\code}[1]{\texttt{#1}}
\newcommand{\file}[1]{\texttt{\bfseries#1}}
\newcommand{\option}[1]{\texttt{\itshape#1}}

%----------------------------------------------------------------------------------------
% Motivation and Background
\section{The need for topic models}
% - motivate topic models
% data is big
Researchers today are faced with a deluge of data. As we continue to digitize and aggregate our collective knowledge we produce ever increasing archives of information. The sheer volume and variety of forms this information may take - text, images, audio, video, social connections etc. - makes it difficult and in most cases impossible to parse manually. 

% need better tool
This driving factor of data growth has given rise to internet giants such as Google, who's search tool helps us access and browse pre-indexed swathes of information. However in order to go beyond mere keyword searches, or link analysis, and break into the realm of understanding each document we need a new approach to data exploration.

% Topic modeling is that tool
A powerful set of computational tools referred to as probabilistic topic models have emerged to meet this challenge. Aimed to discover and annotate large archives of documents with thematic information, topic models identify patterns that reflect the underlying topics which combined to form the documents.

% it can do things unsupervised (so it can automate)
Naturally, it is rare that we would know beforehand exactly what topics a given document contains, and thus topic modeling constitutes an unsupervised task. As a result, topic modeling algorithms are designed to work without prior knowledge of the topic distribution of a given document — that is, the topics are woven from texts themselves. This makes the organization, summarization and annotation of text corpora possible at an inhuman scale. Consequently, topic models are useful in a variety of settings and have successfully been applied to web archives, news articles \parencite{Newman:2006:AET:2106961.2106971}, and academic literature \parencite{Steyvers:2004:PAM:1014052.1014087} to elicit insight. In this paper, focus our experiments on patent data.
% and thus is used in a bunch of stuff (mention patents)
% - explain what they are used for nowadays


%hidden/latent thematic structure in large archives of documents.
%vast quantities of unstructured data
%large databases of text
%electronic document archives
%collections of text

%automatically organizing, searching, indexing, understanding and browsing large collections


%----------------------------------------------------------------------------------------
\section{Latent Dirichilet allocation}
% - explain how they typically work (distribution over topics)
LDA is a probabilistic model
\begin{itemize}
\item Documents can manifest multiple topics (however typically not many)
\item Each document is assumed to be the product of a generative process.
\item Generative process starts with a topic, i.e. a distribution over a fixed vocabulary.
\item Assumes a fixed number of topics
\end{itemize}
its rather intuitive
% include topic highlighted text as an example.

%----------------------------------------------------------------------------------------
\section{Adding a temporal component}
% - describe how a dynamic model might improve on this
% - describe the usefulness of having a temporal component in the model. (maybe give specific example documents)

%----------------------------------------------------------------------------------------
\section{Applying to Patents}
% - reference why applying it to patents is not an accident. i.e. why patents provide an interesting use case.

%----------------------------------------------------------------------------------------
\section{Experiments}
% What we explored:
% - how the word distributions of the topics trended (validate 
% with history)
% - Which models had the 'best' topics as measured by various topic coherence measures
% - Whether the resulting document vectors were useful for document classification
% - Whether the vector space resulted in quality clusters
% -- Usefulness as a measure of influence (f. citations and pagerank) 

% - If you believe all of these things you should give me a distinction. If not, read on...


% typical patent analysis usually includes forward citations as a measure of patent 'value' either economically or technologically.

%----------------------------------------------------------------------------------------

qualitative exploratory tasks and quantitative predictive and classification tasks.



%----------------------------------------------------------------------------------------


\section{examples}
\cite{Blei:2006:DTM:1143844.1143859}
\citep{Chang:Boyd-Graber:Wang:Gerrish:Blei-2009}
\citep{DBLP:journals/corr/RosnerHRNB14}
\citep{DBLP:journals/corr/abs-1206-3298}
\citep{Hall:2008:SHI:1613715.1613763}
\citep{Wang:2006:TOT:1150402.1150450}
\cite{conf/icdm/AlSumaitBD08}
\citep{icml2010_GerrishB10}

Multiple references are separated by semicolons (e.g. \parencite{Blei:2006:DTM:1143844.1143859, Wang:2006:TOT:1150402.1150450}) and 

Scientific references should come \emph{before} the punctuation mark if there is one (such as a comma or period). The same goes for footnotes\footnote{Such as this footnote, here down at the bottom of the page.}.states: \enquote{Footnote numbers should be superscripted, [...], following any punctuation mark except a dash.} The Chicago manual of style states: \enquote{A note number should be placed at the end of a sentence or clause. The number follows any punctuation mark except the dash, which it precedes. It follows a closing parenthesis.}

The bibliography is typeset with references listed in alphabetical order by the first author's last name. This is similar to the APA referencing style. To see how \LaTeX{} typesets the bibliography, have a look at the very end of this document (or just click on the reference number links in in-text citations).


Tables are an important way of displaying your results, below is an example table which was generated with this code:

{\small
\begin{verbatim}
\begin{table}
\caption{The effects of treatments X and Y on the four groups studied.}
\label{tab:treatments}
\centering
\begin{tabular}{l l l}
\toprule
\tabhead{Groups} & \tabhead{Treatment X} & \tabhead{Treatment Y} \\
\midrule
1 & 0.2 & 0.8\\
2 & 0.17 & 0.7\\
3 & 0.24 & 0.75\\
4 & 0.68 & 0.3\\
\bottomrule\\
\end{tabular}
\end{table}
\end{verbatim}
}

\begin{table}
\caption{The effects of treatments X and Y on the four groups studied.}
\label{tab:treatments}
\centering
\begin{tabular}{l l l}
\toprule
\tabhead{Groups} & \tabhead{Treatment X} & \tabhead{Treatment Y} \\
\midrule
1 & 0.2 & 0.8\\
2 & 0.17 & 0.7\\
3 & 0.24 & 0.75\\
4 & 0.68 & 0.3\\
\bottomrule\\
\end{tabular}
\end{table}

You can reference tables with \verb|\ref{<label>}| where the label is defined within the table environment. See \file{Chapter1.tex} for an example of the label and citation (e.g. Table~\ref{tab:treatments}).


\begin{figure}[h]
\centering
\includegraphics{Figures/Electron}
\decoRule
\caption[An Electron]{An electron (artist's impression).}
\label{fig:Electron}
\end{figure}

\begin{equation}
E = mc^{2}
\label{eqn:Einstein}
\end{equation}


\begin{flushright}
Guide written by ---\\
Sunil Patel: \href{http://www.sunilpatel.co.uk}{www.sunilpatel.co.uk}\\
Vel: \href{http://www.LaTeXTemplates.com}{LaTeXTemplates.com}
\end{flushright}
